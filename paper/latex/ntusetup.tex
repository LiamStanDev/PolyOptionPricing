% !TeX root = ./main.tex

% --------------------------------------------------
% 資訊設定(Information Configs)
% --------------------------------------------------

\ntusetup{
  university*   = {National Taiwan University},
  university    = {國立臺灣大學},
  college       = {管理學院},
  college*      = {College of Management},
  institute     = {財務金融學研究所},
  institute*    = {Graduate Institute of Finance},
  title         = {一個高效率且可泛用於多種隨機過程之歐式多項式選擇權定價模型},
  title*        = {An Efficient and General Framework for Pricing European-Style Polynomial Options under Various Stochastic Processes},
  author        = {林大中},
  author*       = {Da-Zhong Lin},
  ID            = {R1023041},
  advisor       = {莊文議, 王之彥},
  advisor*      = {Wen-I Chuang, Jr-Yan Wang},
  date          = {2023-07-01},         % 若註解掉,則預設為當天
  oral-date     = {2023-07-13},         % 若註解掉,則預設為當天
  DOI           = {10.5566/NTU2018XXXXX},
  keywords      = {多項式選擇權, 傅立葉餘弦展開, 選擇權定價, 隨機過程},
  keywords*     = {polynomial options, Fourier cosine expansion, options pricing model, stochastic process},
}

% --------------------------------------------------
% 加載套件(Include Packages)
% --------------------------------------------------

\usepackage[sort&compress]{natbib}      % 參考文獻
\usepackage{amsmath, amsthm, amssymb}   % 數學環境
% \usepackage{ulem, CJKulem}              % 下劃線、雙下劃線與波浪紋效果
% \usepackage{booktabs}                   % 改善表格設置
% \usepackage{multirow}                   % 合併儲存格
% \usepackage{diagbox}                    % 插入表格反斜線
% \usepackage{array}                      % 調整表格高度
% \usepackage{longtable}                  % 支援跨頁長表格
% \usepackage{paralist}                   % 列表環境
\usepackage{graphicx}                   % 圖

\usepackage{lipsum}                     % 英文亂字
\usepackage{zhlipsum}                   % 中文亂字

% --------------------------------------------------
% 套件設定(Packages Settings)
% --------------------------------------------------
\graphicspath{{figures/}}               % 設定圖片的存放路徑
\usepackage{floatrow}                   % 控制浮動對象(如圖片和表格)的排版位置
\usepackage{caption}
% \usepackage{fancyhdr}                   % 加载fancyhdr宏包
% \usepackage[backend=biber, style=apa, sorting=nty]{biblatex}
% \addbibresource{references.bib}
% \pagestyle{fancy}                       % 设置自定义页眉和页脚
% \fancyhf{}                              % 清空页眉页脚
% \fancyhead[L]{\leftmark}                % 在左侧页眉显示当前章标题
% \fancyhead[R]{\thepage}                 % 在页脚中心显示页码
% \renewcommand{\headrulewidth}{0.4pt}    % 设置页眉分割线的宽度为0pt
% \setlength{\intextsep}{12bp \@plus4pt \@minus2pt}


% \captionsetup{}
% \captionsetup[table]{position=top,belowskip={12bp-\intextsep},aboveskip=6bp}
% \captionsetup[figure]{position=below,belowskip={12bp-\intextsep},aboveskip=6bp}
\captionsetup[figure]{position=top}
% \captionsetup[sub]{position=bottom,skip=6bp}
\usepackage[export]{adjustbox}
