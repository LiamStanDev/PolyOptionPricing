% !TeX root = ../main.tex
\chapter{Conclusions}
This study proposes a highly efficient pricing method for polynomial options, leveraging the Fourier method to exploit the relationship between the characteristic function and the coefficients in the Fourier-cosine expansion of the density function. The key advantage of this approach is its ability to separate the impact of the option payoff function and the underlying stochastic process, enabling it to accommodate a wide range of stochastic processes and payoff functions beyond plain vanilla options.

The research findings demonstrate that this pricing model exhibits a remarkably fast error convergence rate and excellent computational efficiency. Moreover, the study presents pricing results for various polynomial options, encompassing both convex and concave payoff functions for the right end. Notably, the proposed method accurately prices these complex polynomial options from first order to fourth order, with the error convergence rate remaining nearly exponential. Furthermore, to ensure the accuracy of pricing, we conducted a large number of Monte Carlo simulations within a highly stringent standard deviation, which corresponds to a very narrow confidence interval. Our pricing model consistently falls within the confidence interval, with the majority of results closely aligning with the mean value.

However, it encounters limitations when dealing with certain complex pricing models with singular points in their characteristic function, such as the stochastic double jumps model. To enhance the applicability of my pricing method and address the issue of singularities, future research could explore alternatives or develop more robust and flexible techniques for approximating the density function. Additionally, further investigation could be conducted to examine the impact of singularities on pricing accuracy and explore potential adjustments or modifications to improve the method's performance in such scenarios.
