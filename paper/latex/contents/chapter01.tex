% !TeX root = ../main.tex

\chapter{Introduction and Literature Review}

In today's highly competitive and sophisticated financial markets, the ability to accurately price options using efficient numerical methods has become increasingly crucial. However, the traditional approach of pricing plain vanilla options is often insufficient to meet the diverse needs of market participants. To address this challenge, researchers are constantly striving to develop pricing techniques that can accommodate a broader range of stochastic processes and more flexible payoff functions.\\

The purpose of this study is to contribute to this ongoing effort by proposing an improved method for assessing polynomial options. By introducing a pricing technique that can handle various stochastic processes and more versatile payoff functions, we aim to provide a single pricing formula that caters to two fundamental aspects of options: the underlying stochastic process and the specific payoff structure. My research not only focuses on enhancing the pricing methodology but also emphasizes the importance of efficiency in option pricing. We have developed a novel approach that exhibits remarkable pricing efficiency with linear complexity and ensures swift calculations. This means that our model can handle complex pricing scenarios in real-time, allowing market participants to make timely and informed decisions. Furthermore, our pricing technique exhibits a high level of accuracy with exponential error convergence. This means that as we increase the computational resources. This ensures that our pricing model provides highly reliable and precise option valuations, even for complex derivative instruments with intricate stochastic processes and flexible payoff functions. The combination of our model's high pricing efficiency and exponential error convergence makes it a valuable tool for market participants, including investors, traders, and risk managers. They can rely on our pricing methodology to swiftly and accurately value a wide array of options, giving them a competitive edge in the fast-paced and highly demanding financial markets. \\

 Black and Scholes' (1973) model assumes that the price process follows geometric Brownian motion, which implies constant volatility. However, empirical evidence has shown that real-world financial data often exhibit characteristics that deviate from this assumption. For instance, returns tend to have an asymmetric leptokurtic distribution with left skew and fat tails, and the implied volatility derived from option prices often exhibits a smile pattern. To address these discrepancies, researchers have proposed various alternative models that can better capture the complexities of financial markets. One influential extension was Merton's (1976) inclusion of jumps in the diffusion model. Jumps represent sudden and significant price movements, which can account for rare events and extreme returns. By combining diffusion and jump processes, Merton's model improved the ability to explain and predict market behavior. Another notable contribution came from Heston (1993), who introduced a simultaneous consideration of the price process and the volatility process. Heston's model allows the volatility to be stochastic and correlated with the underlying asset price. This addition captured the observed phenomenon of volatility clustering, where periods of high volatility tend to be followed by similar periods, and vice versa. Madan and Seneta (1990) proposed the use of the variance gamma process, which incorporates a time-changed Brownian motion, to model asset prices. This approach allows for more flexibility in capturing the distributional characteristics of returns, including skewness and kurtosis. Bates (1996) extended the jump-diffusion model by incorporating stochastic volatility. This innovation accounted for the empirical finding that volatility itself is not constant but rather varies over time. By incorporating stochastic volatility, the model can better capture the dynamics of asset prices and the pricing of options. Barnodorff-Nielsen (1997) introduced the normal inverse Gaussian distribution to describe the distribution of asset returns. This distribution allows for skewness, fat tails, and a wide range of kurtosis values, providing a more realistic representation of the empirical distribution of returns. Kou (2002) suggested a further refinement by splitting jumps into two different exponential distributions. This modification allows for more flexibility in capturing the characteristics of large price movements. While these alternative models offer improvements over the original Black-Scholes framework, they also introduce greater complexity. As a result, obtaining closed-form analytic solutions for option prices becomes increasingly challenging, even for basic plain vanilla options. Researchers often resort to numerical methods or approximation techniques to estimate option prices within these more complex frameworks. \\

 Macovschi and Quittard-Pinon (2006) demonstrated that it is possible to obtain an analytical solution for pricing polynomial options. However, the conditions required for the pricing formula are somewhat counterintuitive and may be challenging to implement in practice. To address this issue, Wang et al. (2022) employed the trinomial lattice method, which is derived from the tree pricing method, to reduce the time complexity of pricing polynomial options to linear and to make it practical for use. However, this method did not effectively reduce the error as the number of time partitions of the tree model increased. In this study, I take a different approach and do not employ tree models or other numerical methods like the finite difference method for pricing, but rather use the Fourier method. \\
 
 The Fourier transform is a mathematical tool that has numerous applications, particularly in the areas of signal processing. It is a well-understood and widely used technique that allows for the representation of a function as a linear combination of sine and cosine functions. Heston (1993) introduced a pioneering approach to options pricing known as the Heston model. This model leverages the Fourier transformation relationship between the density function and the characteristic function, which are Fourier pairs. By utilizing this relationship, Heston derived a semi-closed form solution for the Heston model. Despite the availability of the semi-closed form solution, the Heston model still requires numerical integration to calculate the integral involved in the pricing formula. This numerical integration step contributes to an inevitable quadratic time complexity, which means that the computational time increases with the square of the number of grid points used in the integration process. Additionally, the Heston model is not flexible enough to accommodate complex payoff functions. Carr and Madan (1999) introduced a method that utilizes the Fourier transform of the entire option price, incorporating the specific payoff function, and employs the fast Fourier transform (FFT) algorithm. However, a limitation of the FFT approach is the unavailability of the Fourier transform for the original call option. As a workaround, they propose calculating the Fourier transform of a modified call option price rather than the original call option, which requires a carefully crafted damping factor. Consequently, their method becomes challenging, or even infeasible, and exhibits higher time complexity, loglinear time complexity, when pricing plain vanilla options. In contrast, Lewis (2001) entirely separates the underlying stochastic process from the derivative payoff with the aid of the Plancherel–Parseval Theorem and obtains a variety of valuation formulae by the application of Residual Calculus. However, this method involves using residual calculus and determining the strip, which can be quite challenging. Finally, Fang and Oosterlee (2008) introduced a simpler pricing model called the COS method that fully leverages the connection between the characteristic function and the coefficients in the Fourier-cosine expansion of the density function. This method fully leverages the ability to separate the impact of price dynamics and the payoff function. It does not rely on damping parameters required by the FFT algorithm and effectively handles intractable residual calculus. Most importantly, It even exhibits exponential error convergence with linear time complexity.\\
 
 My research extends the COS method by further generalizing the plain-vanilla payoff function exhibited in their research to a more widely applicable polynomial payoff function and provides a complete formula for valuing call, put, and polynomial options without modifying the method. The effectiveness of the COS method in complex option pricing is also verified and a highly efficient pricing method is provided.\\

 This study is divided into four chapters. Chapter 2 presents my method, which is based on the COS method developed by Fang and Oosterlee (2008) and further derives a general pricing formula for polynomial options. Chapter 3 presents the results, which demonstrate the pricing of different price dynamics based on plain-vanilla and two types of polynomial options - convex payoff for the right-end, and concave payoff for the right-end - using my method and analyzing the error convergence effects. Monte Carlo simulation is also used to ensure the accuracy of the results. Chapter 4 provides the conclusion.\\